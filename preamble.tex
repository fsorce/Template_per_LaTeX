\usepackage[utf8]{inputenc}
\usepackage{amsmath,amssymb,amsfonts,amsthm,stmaryrd}
\usepackage{mathrsfs} % per mathscr
\usepackage{dsfont} % per mathbb1
\usepackage{graphicx}% ruota freccia per le azioni
\usepackage{oldgerm} % Fractur Particolare
\usepackage{marvosym}% per il \Lightning
\usepackage{array}
\usepackage{faktor} %per gli insiemi quoziente
\usepackage{hyperref}
\usepackage{xparse} % Per nuovi comandi con tanti input opzionali
\usepackage{tikz-cd}
\usepackage{multicol}
\usepackage{multirow}
\usepackage{cancel}
\usepackage[italian]{babel}


% Ambienti per teoremi =================================
% <name> 
% <space above> 
% <space below> 
% <body font> 
% <indent amount> 
% <Theorem head font> 
% <punctuation after theorem head> 
% <space after theorem head> (default .5em) 
% <Theorem head spec>

% I nuovi ambienti sono costruiti in modo da andare alla riga successiva
\newtheoremstyle{customth}
{\topsep}{\topsep}{\itshape}{}{\bfseries}{.}{\newline}{}
\newtheoremstyle{customdef}
{\topsep}{\topsep}{\normalfont}{}{\bfseries}{.}{\newline}{}
\newtheoremstyle{customrem}
{\topsep}{\topsep}{\normalfont}{}{\itshape}{.}{\newline}{}

\theoremstyle{customth}
\newtheorem{theorem}{Teorema}[chapter]
\newtheorem{lemma}[theorem]{Lemma}
\newtheorem{corollary}[theorem]{Corollario}
\newtheorem{proposition}[theorem]{Proposizione}
\newtheorem{fact}[theorem]{Fatto}
\newtheorem{application}[theorem]{Applicazione}
\theoremstyle{customrem}
\newtheorem{remark}[theorem]{Osservazione}
\theoremstyle{customdef}
\newtheorem{definition}[theorem]{Definizione}
\newtheorem{notation}[theorem]{Notazione}
\newtheorem{example}[theorem]{Esempio}

\makeatletter
\renewenvironment{proof}[1][\proofname]
{
    \par
    \pushQED{\qed}
    \normalfont \topsep6\p@\@plus6\p@\relax
    \trivlist
    \item[\hskip\labelsep\itshape#1\@addpunct{.}]\mbox{}\\*
}
{
    \popQED\endtrivlist\@endpefalse
}
\makeatother



%============ Simboli standard =================
%----------------- Lettere ---------------------
\newcommand{\A}{\mathbb{A}}
\newcommand{\B}{\mathbb{B}}
\newcommand{\C}{\mathbb{C}}
\newcommand{\D}{\mathbb{D}}
\newcommand{\E}{\mathbb{E}}
\newcommand{\F}{\mathbb{F}}
\newcommand{\G}{\mathbb{G}}
\newcommand{\Hb}{\mathbb{H}}
\newcommand{\I}{\mathbb{I}}
\newcommand{\J}{\mathbb{J}}
\newcommand{\K}{\mathbb{K}}
\newcommand{\Lb}{\mathbb{L}}
\newcommand{\M}{\mathbb{M}}
\newcommand{\N}{\mathbb{N}}
\newcommand{\Ob}{\mathbb{O}}
\newcommand{\Pj}{\mathbb{P}}
\newcommand{\Q}{\mathbb{Q}}
\newcommand{\R}{\mathbb{R}}
\newcommand{\Sb}{\mathbb{S}}
\newcommand{\T}{\mathbb{T}}
\newcommand{\U}{\mathbb{U}}
\newcommand{\V}{\mathbb{V}}
\newcommand{\W}{\mathbb{W}}
\newcommand{\X}{\mathbb{X}}
\newcommand{\Y}{\mathbb{Y}}
\newcommand{\Z}{\mathbb{Z}}

\newcommand{\Ac}{\mathcal{A}}
\newcommand{\Bc}{\mathcal{B}}
\newcommand{\Cc}{\mathcal{C}}
\newcommand{\Dc}{\mathcal{D}}
\newcommand{\Ec}{\mathcal{E}}
\newcommand{\Fc}{\mathcal{F}}
\newcommand{\Gc}{\mathcal{G}}
\newcommand{\Hc}{\mathcal{H}}
\newcommand{\Ic}{\mathcal{I}}
\newcommand{\Jc}{\mathcal{J}}
\newcommand{\Kc}{\mathcal{K}}
\newcommand{\Lc}{\mathcal{L}}
\newcommand{\Mc}{\mathcal{M}}
\newcommand{\Nc}{\mathcal{N}}
\newcommand{\Oc}{\mathcal{O}}
\newcommand{\Pc}{\mathcal{P}}
\newcommand{\Qc}{\mathcal{Q}}
\newcommand{\Rc}{\mathcal{R}}
\newcommand{\Sc}{\mathcal{S}}
\newcommand{\Tc}{\mathcal{T}}
\newcommand{\Uc}{\mathcal{U}}
\newcommand{\Vc}{\mathcal{V}}
\newcommand{\Wc}{\mathcal{W}}
\newcommand{\Xc}{\mathcal{X}}
\newcommand{\Yc}{\mathcal{Y}}
\newcommand{\Zc}{\mathcal{Z}}

\newcommand{\Af}{\mathfrak{A}}
\newcommand{\Bf}{\mathfrak{B}}
\newcommand{\Cf}{\mathfrak{C}}
\newcommand{\Df}{\mathfrak{D}}
\newcommand{\Ef}{\mathfrak{E}}
\newcommand{\Ff}{\mathfrak{F}}
\newcommand{\Gf}{\mathfrak{G}}
\newcommand{\Hf}{\mathfrak{H}}
\newcommand{\If}{\mathfrak{I}}
\newcommand{\Jf}{\mathfrak{J}}
\newcommand{\Kf}{\mathfrak{K}}
\newcommand{\Lf}{\mathfrak{L}}
\newcommand{\Mf}{\mathfrak{M}}
\newcommand{\Nf}{\mathfrak{N}}
\newcommand{\Of}{\mathfrak{O}}
\newcommand{\Pf}{\mathfrak{P}}
\newcommand{\Qf}{\mathfrak{Q}}
\newcommand{\Rf}{\mathfrak{R}}
\newcommand{\Sf}{\mathfrak{S}}
\newcommand{\Tf}{\mathfrak{T}}
\newcommand{\Uf}{\mathfrak{U}}
\newcommand{\Vf}{\mathfrak{V}}
\newcommand{\Wf}{\mathfrak{W}}
\newcommand{\Xf}{\mathfrak{X}}
\newcommand{\Yf}{\mathfrak{Y}}
\newcommand{\Zf}{\mathfrak{Z}}

\newcommand{\af}{\mathfrak{a}}

\newcommand{\cf}{\mathfrak{c}}
\newcommand{\df}{\mathfrak{d}}
\newcommand{\ef}{\mathfrak{e}}
\newcommand{\ff}{\mathfrak{f}}
\newcommand{\gf}{\mathfrak{g}}
\newcommand{\hf}{\mathfrak{h}}

\newcommand{\jf}{\mathfrak{j}}
\newcommand{\kf}{\mathfrak{k}}
\newcommand{\lf}{\mathfrak{l}}
\newcommand{\mf}{\mathfrak{m}}
\newcommand{\nf}{\mathfrak{n}}
\newcommand{\of}{\mathfrak{o}}
\newcommand{\pf}{\mathfrak{p}}
\newcommand{\qf}{\mathfrak{q}}
\newcommand{\rf}{\mathfrak{r}}

\newcommand{\tf}{\mathfrak{t}}
\newcommand{\uf}{\mathfrak{u}}
\newcommand{\vf}{\mathfrak{v}}
\newcommand{\wf}{\mathfrak{w}}
\newcommand{\xf}{\mathfrak{x}}
\newcommand{\yf}{\mathfrak{y}}
\newcommand{\zf}{\mathfrak{z}}

\newcommand{\As}{\mathscr{A}}
\newcommand{\Bs}{\mathscr{B}}
\newcommand{\Cs}{\mathscr{C}}
\newcommand{\Ds}{\mathscr{D}}
\newcommand{\Es}{\mathscr{E}}
\newcommand{\Fs}{\mathscr{F}}
\newcommand{\Gs}{\mathscr{G}}
\newcommand{\Hs}{\mathscr{H}}
\newcommand{\Is}{\mathscr{I}}
\newcommand{\Js}{\mathscr{J}}
\newcommand{\Ks}{\mathscr{K}}
\newcommand{\Ls}{\mathscr{L}}
\newcommand{\Ms}{\mathscr{M}}
\newcommand{\Ns}{\mathscr{N}}
\newcommand{\Os}{\mathscr{O}}
\newcommand{\Ps}{\mathscr{P}}
\newcommand{\Qs}{\mathscr{Q}}
\newcommand{\Rs}{\mathscr{R}}
\newcommand{\Ss}{\mathscr{S}}
\newcommand{\Ts}{\mathscr{T}}
\newcommand{\Us}{\mathscr{U}}
\newcommand{\Vs}{\mathscr{V}}
\newcommand{\Ws}{\mathscr{W}}
\newcommand{\Xs}{\mathscr{X}}
\newcommand{\Ys}{\mathscr{Y}}
\newcommand{\Zs}{\mathscr{Z}}

\newcommand{\ula}{{\underline{a}}}
\newcommand{\ulb}{{\underline{b}}}
\newcommand{\ulc}{{\underline{c}}}
\newcommand{\uld}{{\underline{d}}}
\newcommand{\ule}{{\underline{e}}}
\newcommand{\ulf}{{\underline{f}}}
\newcommand{\ulg}{{\underline{g}}}
\newcommand{\ulh}{{\underline{h}}}
\newcommand{\uli}{{\underline{i}}}
\newcommand{\ulj}{{\underline{j}}}
\newcommand{\ulk}{{\underline{k}}}
\newcommand{\ull}{{\underline{l}}}
\newcommand{\ulm}{{\underline{m}}}
\newcommand{\uln}{{\underline{n}}}
\newcommand{\ulo}{{\underline{o}}}
\newcommand{\ulp}{{\underline{p}}}
\newcommand{\ulq}{{\underline{q}}}
\newcommand{\ulr}{{\underline{r}}}
\newcommand{\uls}{{\underline{s}}}
\newcommand{\ult}{{\underline{t}}}
\newcommand{\ulu}{{\underline{u}}}
\newcommand{\ulv}{{\underline{v}}}
\newcommand{\ulw}{{\underline{w}}}
\newcommand{\ulx}{{\underline{x}}}
\newcommand{\uly}{{\underline{y}}}
\newcommand{\ulz}{{\underline{z}}}

%---------- Funzioni standard ------------------
\newcommand{\Adj}{\mathrm{Adj}\,}
\newcommand{\adj}{\mathrm{adj}\,}
\newcommand{\Ann}{\mathrm{Ann}\,}
\newcommand{\Arg}{\mathrm{Arg}\,}
\newcommand{\Ass}{\mathrm{Ass}\,}
\newcommand{\cha}{\mathrm{char}\,}
\newcommand{\cod}{\mathrm{cod}}
\newcommand{\coker}{\mathrm{coker}\,}
\newcommand{\comb}{\mathrm{Comb}\,}
\newcommand{\dom}{\mathrm{dom}}
\newcommand{\End}{\mathrm{End}\,}
\newcommand{\Fix}{\mathrm{Fix}\;}
\newcommand{\Hom}{\mathrm{Hom}\,}
\newcommand{\imm}{\mathrm{Imm}\,}
\newcommand{\Ind}{\mathrm{Ind}}
\newcommand{\mcd}{\mathrm{mcd}\,}
\newcommand{\mcm}{\mathrm{mcm}\,}
\newcommand{\Min}{\mathrm{Min}\,}
\newcommand{\Mor}{\mathrm{Mor}}
\newcommand{\obj}{\mathrm{obj}}
\newcommand{\orb}{\mathrm{orb}\,}
\newcommand{\ord}{\mathrm{ord}\;}
\newcommand{\Res}{\mathrm{Res}}
\newcommand{\rnk}{\mathrm{rnk}\,}
\newcommand{\sgn}{\mathrm{sgn}\,}
\newcommand{\Span}{\mathrm{Span}\,}
\newcommand{\Spec}{\mathrm{Spec}\,}
\newcommand{\stab}{\mathrm{stab}\,}
\newcommand{\Supp}{\mathrm{Supp}\,}
\newcommand{\supp}{\mathrm{supp}\,}
\newcommand{\tr}{\mathrm{tr}\,}

\newcommand{\Real}{\,\Re\mathfrak{e}}
\newcommand{\Imag}{\,\Im\mathfrak{m}}

%-------------- Frecce -------------------------
\newcommand{\coimplies}{\Longleftrightarrow}
\newcommand{\inj}{\hookrightarrow}
\newcommand{\onto}{\twoheadrightarrow}
\newcommand{\ot}{\leftarrow}
\newcommand{\acts}{\curvearrowright}

%----------- Lettere greche -------------------
\newcommand{\al}{\alpha}
\newcommand{\de}{\delta}
\newcommand{\e}{\varepsilon}
%\newcommand{\th}{\theta}
\newcommand{\la}{\lambda}
\newcommand{\vp}{\varphi}

%-------------- Derivate ----------------------
\newcommand{\raiseargument}[1]{\raisebox{.8ex}{$#1$}}
\newcommand{\centersmallmath}[1]{\vcenter{\hbox{\scalebox{.8}{$#1$}}}}
\newcommand{\raiseargumentsmall}[1]{\raisebox{.4ex}{\scalebox{.8}{$#1$}}}
\newcommand*{\emptyfrac}[2]{\genfrac{}{}{0pt}{}{#1}{#2}}

\NewDocumentCommand{\ddxi}{O{x}mm}{
    {\frac{d^{}{#3}}{d{#1}_{#2}}}
}

\NewDocumentCommand{\dd}{O{}mm}{
    {\frac{d^{#1}{#3}}{d{#2}^{#1}}}
}

\NewDocumentCommand{\ppxi}{O{x}mm}{
    {{\frac{\partial^{}{#3}}{\partial{#1}_{#2}}}}
}

\NewDocumentCommand{\pp}{O{}mm}{
    {{\frac{\partial^{#1}{#3}}{\partial{#2}}}}
}





%========== Comandi dattilografici ============
%--------- Passaggi in derivazioni ------------
\newcommand{\pasg}[3]{\overset{\hyperref[#3]{\text{#2}}}{#1}}
\newcommand{\pasgnl}[2]{\overset{\text{#2}}{#1}}
\newcommand{\pasgnlmath}[2]{\overset{#2}{#1}}
\newcommand{\pasgmath}[3]{\overset{\hyperref[#3]{{#2}}}{#1}}

%----------- Modifica testo -------------------
\newcommand{\ul}[1]{\underline{#1}}
\newcommand{\ol}[1]{\overline{#1}}
\newcommand{\wt}[1]{\widetilde{#1}}
\newcommand{\wh}[1]{\widehat{#1}}
\newcommand{\td}[1]{\Tilde{#1}}
\newcommand{\rg}[1]{{\mathring {#1}}}
\newcommand{\under}[2]{\underset{#1}{\underbrace{#2}}}

%-------------- Parentesi ---------------------
\newcommand{\pa}[1]{\left({#1}\right)}
\newcommand{\spa}[1]{\left[{#1}\right]}
\newcommand{\cpa}[1]{\left\{{#1}\right\}}
\newcommand{\abs}[1]{\left|{#1}\right|}
\newcommand{\norm}[1]{\left\Vert{#1}\right\Vert}
\newcommand{\ps}[1]{\left\langle {#1}\right\rangle}
\newcommand{\floor}[1]{\left\lfloor {#1}\right\rfloor}
\newcommand{\ceil}[1]{\left\lceil {#1}\right\rceil}
\newcommand{\rbar}[1]{\left.{#1}\right|}

%--------------- Matrici ----------------------
\newcommand{\mat}[1]{\begin{pmatrix}#1\end{pmatrix}}
\newcommand{\emat}[1]{\begin{matrix}#1\end{matrix}}
\newcommand{\dmat}[1]{\begin{vmatrix}#1\end{vmatrix}}
\newcommand{\smat}[1]{\begin{smallmatrix}#1\end{smallmatrix}}
\newcommand{\BIG}[1]{\mathlarger{\mathlarger{\mathlarger{\mathlarger{#1}}}}}

%--------------- Funzioni ---------------------
\newcommand{\funcDef}[4]{
\begin{array}{ccc}
{#1} & \longrightarrow & {#2}\\
{#3} & \longmapsto & {#4}
\end{array}}

%---------------- Altro -----------------------
\newcommand{\bs}{\setminus}
\newcommand{\res}[1]{\raisebox{-.5ex}{$|$}_{#1}}
\newcommand{\quot}[2]{\faktor{#1}{#2}}
\newcommand{\sep}{\,\middle|\,}

\newcommand{\ii}{^{-1}}
\newcommand{\nz}{\bs\{0\}}

\newcommand{\powerset}{\mathscr{P}}
\newcommand{\del}{\partial}
\newcommand{\0}{{\underline{0}}}
\newcommand{\1}{{\vcenter{\hbox{\scalebox{1.2}{$\mathds{1}$}}}}}

\NewDocumentCommand{\PGL}{o m}{
    \IfNoValueTF{#1}
        {{\mathbb{P}GL({#2})}}
    {{\mathbb{P}GL_{#1}({#2})}}
}
\NewDocumentCommand{\GL}{o m}{
    \IfNoValueTF{#1}
        {{GL({#2})}}
    {{GL_{#1}({#2})}}
}
\newcommand{\znz}[1]{{\Z/{#1}\Z}}






%========= Preambolo per quiver ================
% quiver e' uno strumento che uso spesso per
% disegnare diagrammi. L'interfaccia sul loro sito
% permette di creare in modo visivo il diagramma e
% poi esportarlo come codice LaTeX da inserire nel
% documento. Il sito e' https://q.uiver.app/ 

%-----------------------------------------------
% *** quiver ***
% A package for drawing commutative diagrams exported from https://q.uiver.app.
%
% This package is currently a wrapper around the `tikz-cd` package, importing necessary TikZ
% libraries, and defining a new TikZ style for curves of a fixed height.
%
% Version: 1.2.1
% Authors:
% - varkor (https://github.com/varkor)
% - Andr\e'C (https://tex.stackexchange.com/users/138900/andr%C3%A9c)

\NeedsTeXFormat{LaTeX2e}
%\ProvidesPackage{quiver}[2021/01/11 quiver]

% `tikz-cd` is necessary to draw commutative diagrams.
\RequirePackage{tikz-cd}
% `amssymb` is necessary for `\lrcorner` and `\ulcorner`.
\RequirePackage{amssymb}
% `calc` is necessary to draw curved arrows.
\usetikzlibrary{calc}
% `pathmorphing` is necessary to draw squiggly arrows.
\usetikzlibrary{decorations.pathmorphing}

% A TikZ style for curved arrows of a fixed height, due to Andr\e'C.
\tikzset{curve/.style={settings={#1},to path={(\tikztostart)
    .. controls ($(\tikztostart)!\pv{pos}!(\tikztotarget)!\pv{height}!270:(\tikztotarget)$)
    and ($(\tikztostart)!1-\pv{pos}!(\tikztotarget)!\pv{height}!270:(\tikztotarget)$)
    .. (\tikztotarget)\tikztonodes}},
    settings/.code={\tikzset{quiver/.cd,#1}
        \def\pv##1{\pgfkeysvalueof{/tikz/quiver/##1}}},
    quiver/.cd,pos/.initial=0.35,height/.initial=0}

% TikZ arrowhead/tail styles.
\tikzset{tail reversed/.code={\pgfsetarrowsstart{tikzcd to}}}
\tikzset{2tail/.code={\pgfsetarrowsstart{Implies[reversed]}}}
\tikzset{2tail reversed/.code={\pgfsetarrowsstart{Implies}}}
% TikZ arrow styles.
\tikzset{no body/.style={/tikz/dash pattern=on 0 off 1mm}}
%=================================================